\section{Manual Installation}
The application is built using MAUI that uses packaged compilation using MSIX.
To change the the target to unpackaged application please follow the steps bellow.
Note that unpackaged publishing is not officially supported and may result in unexpected behaviour.

\subsection{Requirements}
To compile your project in Visual Studio 2022, you will need the following:

\begin{enumerate}
	\item Windows operating system
	\item Visual Studio 2022 installed on your machine
	\item .NET 7 SDK
	\item WinUI 3 SDK
	\item MAUI Visual Studio extension installed
	\item Microsoft Visual Studio Installer Projects extension installed
	\item PowerShell installed on your machine
	\item pki package installed in PowerShell
	\item dotnet package installed in PowerShell
\end{enumerate}

\subsection{Unpackaged compilation}
Add following lines to \textbf{CalculatorDesktopApp.csproj} file into the \textbf{PropertyGroup}~node:

\begin{lstlisting}[language=XML]
<WindowsPackageType>None</WindowsPackageType>
<WindowsAppSDKSelfContained Condition="'$(IsUnpackaged)' == 'true'">true</WindowsAppSDKSelfContained>
<SelfContained Condition="'$(IsUnpackaged)' == 'true'">true</SelfContained> 
\end{lstlisting}
~\\
Change your \textbf{launchSettings.json} file to look like this:
\newline
\begin{lstlisting}[language=json]
{
	"profiles": {
		"Windows Machine": {
			"commandName":~"Project"
		}
	}
}
\end{lstlisting}
~\\
To build the application open \textbf{IvsProject.sln} file in Visual Studio 2022. Select the \textbf{CalculatorDesktopApp} project as startup project and click "Build".

\subsection{Packaged compilation}

\subsection*{Create a certificate}

Open a PowerShell terminal and navigate to the directory with your project.
\\\\
Use the \textbf{New-SelfSignedCertificate} command to generate a self-signed certificate.
\\
The \textbf{PublisherName} value is displayed to the user when they install your app, supply your own value and omit the "()" characters. You can set the \textbf{Name} parameter to any string of text you want.

\begin{lstlisting}[language=XML]
New-SelfSignedCertificate -Type Custom -Subject "CN=(PublisherName)"
-KeyUsage DigitalSignature
-FriendlyName "Name"
-CertStoreLocation "Cert:\CurrentUser\My" `
-TextExtension @("2.5.29.37={text}1.3.6.1.5.5.7.3.3", "2.5.29.19={text}")
\end{lstlisting}
~\\
Use the following PowerShell command to query the certificate store for the certificate that was created:

\begin{lstlisting}[language=XML]
Get-ChildItem "Cert:\CurrentUser\My" | Format-Table Subject, FriendlyName, Thumbprint
\end{lstlisting}

\subsection*{Compile and publish the application}

Navigate to the folder with \textbf{CalculatorDesktopApp} project in PowerShell and use the following command to build the project. Replace "(cert)" with your certificate.
\\\\
\texttt{
dotnet publish -f net7.0-windows10.0.19041.0 -c Release\\ -p:RuntimeIdentifierOverride=win10-x64 -p:PackageCertificateThumbprint=(cert)
}
\\
You should find the path to the packaged application in the terminal path.

